%Copyright 2014 Jean-Philippe Eisenbarth
%This program is free software: you can 
%redistribute it and/or modify it under the terms of the GNU General Public 
%License as published by the Free Software Foundation, either version 3 of the 
%License, or (at your option) any later version.
%This program is distributed in the hope that it will be useful,but WITHOUT ANY 
%WARRANTY; without even the implied warranty of MERCHANTABILITY or FITNESS FOR A 
%PARTICULAR PURPOSE. See the GNU General Public License for more details.
%You should have received a copy of the GNU General Public License along with 
%this program.  If not, see <http://www.gnu.org/licenses/>.

%Based on the code of Yiannis Lazarides
%http://tex.stackexchange.com/questions/42602/software-requirements-specification-with-latex
%http://tex.stackexchange.com/users/963/yiannis-lazarides
%Also based on the template of Karl E. Wiegers
%http://www.se.rit.edu/~emad/teaching/slides/srs_template_sep14.pdf
%http://karlwiegers.com
\documentclass{scrreprt}
\usepackage{listings}
\usepackage{underscore}
\usepackage[bookmarks=true]{hyperref}
\usepackage[utf8]{inputenc}
\usepackage[spanish]{babel}
\hypersetup{
    bookmarks=false,    % show bookmarks bar?
    pdftitle={Especificación de requerimientos de software},    % title
    pdfauthor={Sebastián Maldonado},                     % author
    pdfsubject={Software Requerimientos Especificación},                        % subject of the document
    pdfkeywords={TeX, LaTeX, graphics, images}, % list of keywords
    colorlinks=true,       % false: boxed links; true: colored links
    linkcolor=blue,       % color of internal links
    citecolor=black,       % color of links to bibliography
    filecolor=black,        % color of file links
    urlcolor=purple,        % color of external links
    linktoc=page            % only page is linked
}%
%Definición de la versión del documento.
\def\myversion{1.0 }
\date{}
%\title
\usepackage{hyperref}
\begin{document}

\begin{flushright}
    \rule{16cm}{5pt}\vskip1cm
    \begin{bfseries}
        \Huge{Especificación de Requerimientos de Software}\\
        \vspace{1.9cm}
        \vspace{1.9cm}
        $<$Proyecto$>$\\
        \vspace{1.9cm}
        \LARGE{Versión \myversion aprobada}\\
        \vspace{1.9cm}
        Escrito por $<$autor$>$\\
        \vspace{1.9cm}
        $<$Organización$>$\\
        \vspace{1.9cm}
        \today\\
    \end{bfseries}
\end{flushright}

\tableofcontents


\chapter*{Historial de Cambios}

\begin{center}
    \begin{tabular}{|c|c|c|c|}
        \hline
	    Nombre & Fecha & Razón del cambio & Versión\\
        \hline
	    21 & 22 & 23 & 24\\
        \hline
	    31 & 32 & 33 & 34\\
        \hline
    \end{tabular}
\end{center}

\chapter{Introducción}

\section{Propósito}
$<$Identificar el producto para que se están tomando los requerimientos del documento, incluyendo la revisión y el número de release en caso de aplicar.
Describe el alcance del producto, particularmente si este SRS describe solo una parte del sistema o un solo sub sistema.$>$

\section{Convenciones del documento}
$<$Describe cualquier estándar o convención tipográfica que fue seguida para la confección del documento SRS, así como tipos de fuentes, subrayados o formas de resaltar lo escrito que tenga alguna significancia especial.$>$

\section{Para quién está dirigido el documento}
$<$Describe los diferentes tipos de lectores para los cuales está dirigido el documento, como por ejemplo, desarrolladores, jefes de proyecto, usuarios finales, testers, marketing, documentadores técnicos, etc.
Además debería sugerir una secuencia para ser leído el documento por cada tipo de usuario donde al que está dirigido, indicando las partes que pueden ser relevantes según el perfil.$>$

\section{Alcance del proyecto}
$<$Provee una descripción corta del software descrito y su propósito, incluyendo los beneficios relevantes, objetivos y metas.
Esta sección debe indicar como el software se relaciona con los objetivos corporativos y las metas estratégicas de la empresa.
Se debe incorporar la visión y misión del proyecto, si está está disponible en otro documento, debe hacer referencia al documento el lugar de repetir la información$>$

\section{Referencias}
$<$Lista cualquier otro documento o dirección web al que el SRS haga alusión.
Estas pueden incluir guías de interfaces de usuario, contratos, estándares, documentos de requerimientos adicionales, casos de uso, notas del proyecto, misión y visión, etc.
Se debe proveer la información suficiente para que cualquier lector pueda acceder a una copia de la referencia, incluyendo título, autor, versión, fecha, ubicación y cualquier otra información relevante para desambiguar o poder acceder a los documentos referido.$>$


\chapter{Descripción General}

\section{Perspectiva del producto}
$<$Describe el contexto y origen del producto descrito en el SRS.
Por ejemplo, indica si es un producto nuevo, un derivado de otro producto, reemplazo de un producto, sistema o sub-sistema existente. 
Además de lo mencionado anteriormente, en esta sección se debe indicar si corresponde a un módulo de un sistema mayor, un sub proyecto dentro del contexto de otro proyecto, y los sistemas colaboradores del mismo proyecto, indicando las interfaces para comunicarse.
En la medida de lo posible esta sección debe estar acompañada de apoyo de figuras y gráficos para poder aclarar donde está el sistema coexistiendo.$>$

\section{Funcionalidades del producto}
$<$Resumen de las principales funcionalidades que el producto debe realizar o los procesos que el producto soporta.
Los detalles de cada funcionalidad deben ir en la sección 3 por lo que solo se debe hacer un resumen de alto nivel.
De preferencia se debe mostrar como las funcionalidades interaccionan entre si, para esto usar de preferencia un diagrama de procesos o un diagrama de flujo de datos.$>$

\section{Tipos de usuarios y características}
$<$Identifica los distintos tipos de usuarios que utilizarán el producto. Hay que identificar para los distintos tipos de usuarios las siguientes características si es que son relevantes, frecuencia de uso, conjunto de funciones utilizadas, conocimiento técnico, seguridad o nivel de privilegios que tiene, nivel educacional o experiencia y características adicionales.Algunas funcionalidades solo aplicarán para algunos tipos de usuarios. Se debe distinguir el nivel de relevancia de los usuarios, así, en caso de haber conflicto, se tiene una herramienta que establece una relevancia y una prioridad a la hora de satisfacer los resultados.$>$

\section{Ecosistema de la aplicación}
$<$Describe el ambiente en el cual la aplicación funcionará, incluyendo hardware, sistema operativo, versión y cualquier otro componente de sistema operativo o aplicación con la que debe coexistir el software.$>$

\section{Restricciones de diseño e implementación}
$<$Describe cualquier elemento o problema que pueda limitar las opciones disponibles para los desarrolladores. Esto puede incluir: políticas coporativas, limitaciones de hardware (requerimiento de tiempos de ejecución o utilización de memoria), interfaces a otras aplicaciones, especificaciones de tecnología, herramientas, base de datos a ser utilizadas, operaciones paralelas, lenguajes requeridos, protocolos de comunicación, consideraciones de seguridad, convenciones y estándares de programación.$>$

\section{Documentación para el usuario}
$<$Lista de la documentación de los componentes para el usuario que serán entregados con el software, identificando el estándar que será utilizado para entregar esta documentación.$>$
\section{Suposiciones y dependencias}
$<$Se lista cualquier suposición que se realice que pueda afectar a los requerimientos levantados en el SRS. Esto puede incluir dependencias de terceras partes o componentes comerciales que se quieran utilizar, problemas y comentarios que puedan surgir de las restricciones técnicas. El proyecto puede verse afectado si las suposiciones son incorrectas, no son compartidas o no se modifican en caso de necesitar cambio. También se debe identificar cualquier factor externo al proyecto que pueda afectar al correcto desempeño del proyecto y si el proyecto o componentes del mismo serán utilizados en otros módulos o proyectos.$>$

\chapter{Requerimientos de interfaces externas.}

\section{Interfaces de usuario}
$<$Describe las características de cada interfaz expuesta a utilización por los usuarios. Este puede incluir ejemplos de GUI, pantallazos de aplicaciones similares, estándares de diseño o guías de estilo de diseño que se deben seguir, restricciones de distribución de pantalla, botones y elementos que se repetirán en la aplicación, shortcut de teclado, despliegue de mensajes estándar al usuario, etc. Definición de los componentes de interfaz de usuario que sean necesarios. Detalle de la interfaz de usuario debería ir en un documento aparte de especificación de interfaz, pero ser referenciado en este punto.$>$

\section{Interfaces de Hardware}
$<$Describe las características físicas y lógicas de cada una de las interfaces de los módulos del software y el hardware del producto. Esto puede incluir los tipos de dispositivos soportados, la naturaleza de los datos que se mueven entre el software y el hardware, los controles y puntos en los que ínter actúan y como el protocolo que utilizarán.$>$ 

\section{Interacciones de software}
$<$Describe las interacciones entre este producto y otro componente de software específico, (nombre y versión), incluyendo bases de datos, sistemas operativos, herramientas, librerías, y componentes comerciales integrados. Se identifican los datos o mensajes entrantes y salientes del sistema y se describe el propósito de estos. Se describen los servicios necesarios y la naturaleza de la comunicación. Se referencian documentos que describan en detalle los protocolos de comunicación utilizados, así como las convenciones. Se identifican los datos que serán compartidos entre los componentes de software. Se especifica y define el método de comunicación de los datos compartidos $($por ejemplo, el uso de un área global de memoria compartida en procesos multitarea$)$, especificando las restricciones de implementación.$>$

\section{Interfaces de comunicación}
$<$Describe los requerimientos asociados con cualquier tipo de comunicación que requerida por el producto, incluyendo e-mail, navegadores web, protocolos de comunicación con otros sistemas, entre otros. Define cualquier formato de comunicación. Identifica cualquier estándar de comunicación utilizado $($ftp, http, etc.$)$. Especifica restricciones y condiciones para realizar la comunicación, por ejemplo sistemas de seguridad, encriptación, ratios de transferencia de datos, tipo de sincronismo, tiempos de respuesta.$>$

\chapter{Funcionalidades del sistema}
$<$Este template muestra la organización de requerimientos funcionales de acuerdo a los principales servicios prestados por el producto. Se puede organizar esta sección por casos de uso, modelo de operación, clases de usuario, objetos del sistema, herencia de funcionalidades o una combinación de alternativas mencionadas, lo que sea más coherente de utilizar.$>$

\section{Funcionalidad del sistema 1}
$<$Hay que utilizar una frase descriptiva de la funcionalidad a ser descrita, utilizando solo algunas palabras, no es necesario enumerar.$>$

\subsection{Descripción y prioridad}
$<$Provee una descripción corta de la funcionalidad y establece su nivel de prioridad, se pueden incorporar componentes adicionales que ayudan a establecer mejor la prioridad como beneficios, costos, riesgos, entre otros.$>$

\subsection{Secuencia de estimulo respuesta}
$<$Lista la secuencia de acciones de usuario y respuestas del sistema que permite definir el comportamiento que debería tener la funcionalidad. Esto debe estar relacionado con los elementos de diálogo de los casos de uso.$>$

\subsection{Requerimientos funcionales}
$<$Detalle los elementos asociados a este requerimiento funcional. Esto es, todos los elementos y capacidades que debe tener el software para llevar a cabo los servicios que permiten al usuario la activad para la que se está construyendo. Hay que incluir como el sistema debe responder a entradas incorrectas errores y las acciones que debe realizar para corregirlos. Los requerimientos deben ser concisos, completos, sin ambigüedades, verificables y necesarios. En caso de no contar con los elementos completos de la información se debe indicar que debe ser completado. Como parte del formato se exige que cada requerimiento debe tener un identificador único para identificarlo del resto como REQ-1 o similares.$>$

\section{Funcionalidad del sistema 2 (continuar)}


\chapter{Otros requerimientos no funcionales}

\section{Requerimientos de desempeño}
$<$Si hay requerimientos de desempeño se deben explicar las razones de los mismos,
así se puede obtener un mejor diseño del producto. Especificar las relaciones del 
tiempo para los productos que requirieren interacción en tiempo real. Hacer estos
requerimientos tan específicos como sea posible. Los requerimientos de desempeño
deberían ser establecidos por cada funcionalidad requerida.$>$

\section{Requerimientos de resguardo}
$<$Especifica los requerimientos que tienen relación con posibles pérdidas, daños
materiales o físicos que pueda provocar el uso del producto. Define cualquier acción
o precaución que pueda ser tomada, así como las acciones que se deben evitar.
Hay que referir cualquier regulación o política externa que afecte al diseño del
producto o a su utilización. Se define cualquier certificación de seguridad que
debe ser satisfecha.$>$

\section{Requerimientos de seguridad}
$<$Especifica cualquier requerimiento que tenga relación con la seguridad o privacidad
del producto por su utilización o la protección de los datos utilizados o creados
por el producto. Define cualquier requerimiento de autenticación requerido por el
producto. Se debe hacer referencia a cualquier política de seguridad o regulaciones
que puedan afectar el producto. Define cualquier certificación de seguridad o
privacidad que deba ser satisfecha.$>$

\section{Requerimientos de calidad de software}
$<$Especifica cualquier característica adicional para el producto que sea importante
tanto para el usuario como para los desarrolladores. Algunas de las que se deben 
considerar son las siguientes: adaptabilidad, disponibilidad, correctitud, 
flexibilidad, interoperatividad, mantenibilidad, portabilidad, confiabilidad, 
reusabilidad, robustes, testeabilidad y usabilidad. Escriba esto para que sea especifico, 
cuantitativo y verificable cuando sea posible. Finalmente, clarifique la preponderancia
de algunos atributos sobre otros, como fácil de utilizar, sobre fácil de aprender.$>$

\section{Reglas de negocio}
$<$Lista cualquier principio de operación para el producto, así como los individuos
o roles que pueden realizar las distintas funciones y en que circunstancias. Estos
no son requerimientos funcionales, pero pueden implicar ciertos requerimientos
funcionales utilizados para hacer cumplir las reglas.$>$

\chapter{Otros requerimientos}
$<$Define cualquier otro requerimiento no cubierto en cualquier otro punto del SRS.
Esto puede incluir requerimientos de bases de datos, requerimientos de internacionalización
o localización, requerimientos legales o sociales, objetivos de reutilización del
proyecto, o cualquiera que no esté listado. Añadir las secciones que sean pertinentes
al proyecto.$>$ 

\section{Apéndice A: Glosario}
%ver https://en.wikibooks.org/wiki/LaTeX/Glossary
$<$Define todos los términos necesarios para la correcta interpretación del SRS, 
incluyendo acrónimos y abreviaciones. Si se desea, se puede realizar un glosario
separado para abarcar múltiples proyectos o la organización completa, y en esta 
sección solo definir los términos específicos para el proyecto, referenciando el 
mencionado diccionario adicional.$>$

\section{Apéndice B: Modelos de análisis}
$<$Opcionalmente se puede incluir cualquier modelo utilizado para al análisis de
las necesidades, como diagramas de procesos, diagramas de flujos, diagramas de estados,
diagramas de clases, diagramas de entidad relación, etc.$>$

\section{Apéndice C: Lista para ser definida}
$<$Mantiene el listado de todos los términos, reglas, usos o funcionalidades que
faltan por definir, para realizar el seguimiento y que puedan ser cerradas.$>$

\end{document}
