%Copyright 2014 Jean-Philippe Eisenbarth
%This program is free software: you can 
%redistribute it and/or modify it under the terms of the GNU General Public 
%License as published by the Free Software Foundation, either version 3 of the 
%License, or (at your option) any later version.
%This program is distributed in the hope that it will be useful,but WITHOUT ANY 
%WARRANTY; without even the implied warranty of MERCHANTABILITY or FITNESS FOR A 
%PARTICULAR PURPOSE. See the GNU General Public License for more details.
%You should have received a copy of the GNU General Public License along with 
%this program.  If not, see <http://www.gnu.org/licenses/>.

%Based on the code of Yiannis Lazarides
%http://tex.stackexchange.com/questions/42602/software-requirements-specification-with-latex
%http://tex.stackexchange.com/users/963/yiannis-lazarides
%Also based on the template of Karl E. Wiegers
%http://www.se.rit.edu/~emad/teaching/slides/srs_template_sep14.pdf
%http://karlwiegers.com
\documentclass{scrreprt}
\usepackage{listings}
\usepackage{underscore}
\usepackage[bookmarks=true]{hyperref}
\usepackage[utf8]{inputenc}
\usepackage[spanish]{babel}
\hypersetup{
    bookmarks=false,    % show bookmarks bar?
    pdftitle={Especificación de requerimientos de software},    % title
    pdfauthor={Sebastián Maldonado},                     % author
    pdfsubject={Software Requerimientos Especificación},                        % subject of the document
    pdfkeywords={TeX, LaTeX, graphics, images}, % list of keywords
    colorlinks=true,       % false: boxed links; true: colored links
    linkcolor=blue,       % color of internal links
    citecolor=black,       % color of links to bibliography
    filecolor=black,        % color of file links
    urlcolor=purple,        % color of external links
    linktoc=page            % only page is linked
}%
%Definición de la versión del documento.
\def\myversion{1.0 }
\date{}
%\title
\usepackage{hyperref}
\begin{document}

\begin{flushright}
    \rule{16cm}{5pt}\vskip1cm
    \begin{bfseries}
        \Huge{Especificación de Requerimientos de Software}\\
        \vspace{1.9cm}
        \vspace{1.9cm}
        $<$Proyecto$>$\\
        \vspace{1.9cm}
        \LARGE{Versión \myversion aprobada}\\
        \vspace{1.9cm}
        Escrito por $<$autor$>$\\
        \vspace{1.9cm}
        $<$Organización$>$\\
        \vspace{1.9cm}
        \today\\
    \end{bfseries}
\end{flushright}

\tableofcontents


\chapter*{Historial de Cambios}

\begin{center}
    \begin{tabular}{|c|c|c|c|}
        \hline
	    Nombre & Fecha & Razón del cambio & Versión\\
        \hline
	    21 & 22 & 23 & 24\\
        \hline
	    31 & 32 & 33 & 34\\
        \hline
    \end{tabular}
\end{center}

\chapter{Introducción}

\section{Propósito}
$<$Identificar el producto para que se están tomando los requerimientos del documento, incluyendo la revisión y el número de release en caso de aplicar.
Describe el alcance del producto, particularmente si este SRS describe solo una parte del sistema o un solo sub sistema.$>$

\section{Convenciones del documento}
$<$Describe cualquier estándar o convención tipográfica que fue seguida para la confección del documento SRS, así como tipos de fuentes, subrayados o formas de resaltar lo escrito que tenga alguna significancia especial.$>$

\section{Para quién está dirigido el documento}
$<$Describe los diferentes tipos de lectores para los cuales está dirigido el documento, como por ejemplo, desarrolladores, jefes de proyecto, usuarios finales, testers, marketing, documentadores técnicos, etc.
Además debería sugerir una secuencia para ser leído el documento por cada tipo de usuario donde al que está dirigido, indicando las partes que pueden ser relevantes según el perfil.$>$

\section{Alcance del proyecto}
$<$Provee una descripción corta del software descrito y su propósito, incluyendo los beneficios relevantes, objetivos y metas.
Esta sección debe indicar como el software se relaciona con los objetivos corporativos y las metas estratégicas de la empresa.
Se debe incorporar la visión y misión del proyecto, si está está disponible en otro documento, debe hacer referencia al documento el lugar de repetir la información$>$

\section{Referencias}
$<$Lista cualquier otro documento o dirección web al que el SRS haga alusión.
Estas pueden incluir guías de interfaces de usuario, contratos, estándares, documentos de requerimientos adicionales, casos de uso, notas del proyecto, misión y visión, etc.
Se debe proveer la información suficiente para que cualquier lector pueda acceder a una copia de la referencia, incluyendo título, autor, versión, fecha, ubicación y cualquier otra información relevante para desambiguar o poder acceder a los documentos referido.$>$


\chapter{Descripción General}

\section{Perspectiva del producto}
$<$Describe el contexto y origen del producto descrito en el SRS.
Por ejemplo, indica si es un producto nuevo, un derivado de otro producto, reemplazo de un producto, sistema o sub-sistema existente. 
Además de lo mencionado anteriormente, en esta sección se debe indicar si corresponde a un módulo de un sistema mayor, un sub proyecto dentro del contexto de otro proyecto, y los sistemas colaboradores del mismo proyecto, indicando las interfaces para comunicarse.
En la medida de lo posible esta sección debe estar acompañada de apoyo de figuras y gráficos para poder aclarar donde está el sistema coexistiendo.$>$

\section{Funcionalidades del producto}
$<$Resumen de las principales funcionalidades que el producto debe realizar o los procesos que el producto soporta.
Los detalles de cada funcionalidad deben ir en la sección 3 por lo que solo se debe hacer un resumen de alto nivel.
De preferencia se debe mostrar como las funcionalidades interaccionan entre si, para esto usar de preferencia un diagrama de procesos o un diagrama de flujo de datos.$>$

\section{Tipos de usuarios y características}
$<$Identifica los distintos tipos de usuarios que utilizarán el producto. Hay que identificar para los distintos tipos de usuarios las siguientes características si es que son relevantes, frecuencia de uso, conjunto de funciones utilizadas, conocimiento técnico, seguridad o nivel de privilegios que tiene, nivel educacional o experiencia y características adicionales.Algunas funcionalidades solo aplicarán para algunos tipos de usuarios. Se debe distinguir el nivel de relevancia de los usuarios, así, en caso de haber conflicto, se tiene una herramienta que establece una relevancia y una prioridad a la hora de satisfacer los resultados.$>$

\section{Ecosistema de la aplicación}
$<$Describe el ambiente en el cual la aplicación funcionará, incluyendo hardware, sistema operativo, versión y cualquier otro componente de sistema operativo o aplicación con la que debe coexistir el software.$>$

\section{Restricciones de diseño e implementación}
$<$Describe cualquier elemento o problema que pueda limitar las opciones disponibles para los desarrolladores. Esto puede incluir: políticas coporativas, limitaciones de hardware (requerimiento de tiempos de ejecución o utilización de memoria), interfaces a otras aplicaciones, especificaciones de tecnología, herramientas, base de datos a ser utilizadas, operaciones paralelas, lenguajes requeridos, protocolos de comunicación, consideraciones de seguridad, convenciones y estándares de programación.$>$

\section{Documentación para el usuario}
$<$Lista de la documentación de los componentes para el usuario que serán entregados con el software, identificando el estándar que será utilizado para entregar esta documentación.$>$
\section{Suposiciones y dependencias}
$<$Se lista cualquier suposición que se realice que pueda afectar a los requerimientos levantados en el SRS. Esto puede incluir dependencias de terceras partes o componentes comerciales que se quieran utilizar, problemas y comentarios que puedan surgir de las restricciones técnicas. El proyecto puede verse afectado si las suposiciones son incorrectas, no son compartidas o no se modifican en caso de necesitar cambio. También se debe identificar cualquier factor externo al proyecto que pueda afectar al correcto desempeño del proyecto y si el proyecto o componentes del mismo serán utilizados en otros módulos o proyectos.$>$

\chapter{Requerimientos de interfaces externas.}

\section{Interfaces de usuario}
$<$Describe las características de cada interfaz expuesta a utilización por los usuarios. Este puede incluir ejemplos de GUI, pantallazos de aplicaciones similares, estándares de diseño o guías de estilo de diseño que se deben seguir, restricciones de distribución de pantalla, botones y elementos que se repetirán en la aplicación, shortcut de teclado, despliegue de mensajes estándar al usuario, etc. Definición de los componentes de interfaz de usuario que sean necesarios. Detalle de la interfaz de usuario debería ir en un documento aparte de especificación de interfaz, pero ser referenciado en este punto.$>$

\section{Interfaces de Hardware}
$<$Describe las características físicas y lógicas de cada una de las interfaces de los módulos del software y el hardware del producto. Esto puede incluir los tipos de dispositivos soportados, la naturaleza de los datos que se mueven entre el software y el hardware, los controles y puntos en los que ínter actúan y como el protocolo que utilizarán.$>$ 

\section{Software Interfaces}
$<$Describe the connections between this product and other specific software 
components (name and version), including databases, operating systems, tools, 
libraries, and integrated commercial components. Identify the data items or 
messages coming into the system and going out and describe the purpose of each.  
Describe the services needed and the nature of communications. Refer to 
documents that describe detailed application programming interface protocols.  
Identify data that will be shared across software components. If the data 
sharing mechanism must be implemented in a specific way (for example, use of a 
global data area in a multitasking operating system), specify this as an 
implementation constraint.$>$

\section{Communications Interfaces}
$<$Describe the requirements associated with any communications functions 
required by this product, including e-mail, web browser, network server 
communications protocols, electronic forms, and so on. Define any pertinent 
message formatting. Identify any communication standards that will be used, such 
as FTP or HTTP. Specify any communication security or encryption issues, data 
transfer rates, and synchronization mechanisms.$>$


\chapter{System Features}
$<$This template illustrates organizing the functional requirements for the 
product by system features, the major services provided by the product. You may 
prefer to organize this section by use case, mode of operation, user class, 
object class, functional hierarchy, or combinations of these, whatever makes the 
most logical sense for your product.$>$

\section{System Feature 1}
$<$Don’t really say “System Feature 1.” State the feature name in just a few 
words.$>$

\subsection{Description and Priority}
$<$Provide a short description of the feature and indicate whether it is of 
High, Medium, or Low priority. You could also include specific priority 
component ratings, such as benefit, penalty, cost, and risk (each rated on a 
relative scale from a low of 1 to a high of 9).$>$

\subsection{Stimulus/Response Sequences}
$<$List the sequences of user actions and system responses that stimulate the 
behavior defined for this feature. These will correspond to the dialog elements 
associated with use cases.$>$

\subsection{Functional Requirements}
$<$Itemize the detailed functional requirements associated with this feature.  
These are the software capabilities that must be present in order for the user 
to carry out the services provided by the feature, or to execute the use case.  
Include how the product should respond to anticipated error conditions or 
invalid inputs. Requirements should be concise, complete, unambiguous, 
verifiable, and necessary. Use “TBD” as a placeholder to indicate when necessary 
information is not yet available.$>$

$<$Each requirement should be uniquely identified with a sequence number or a 
meaningful tag of some kind.$>$

REQ-1:	REQ-2:

\section{System Feature 2 (and so on)}


\chapter{Other Nonfunctional Requirements}

\section{Performance Requirements}
$<$If there are performance requirements for the product under various 
circumstances, state them here and explain their rationale, to help the 
developers understand the intent and make suitable design choices. Specify the 
timing relationships for real time systems. Make such requirements as specific 
as possible. You may need to state performance requirements for individual 
functional requirements or features.$>$

\section{Safety Requirements}
$<$Specify those requirements that are concerned with possible loss, damage, or 
harm that could result from the use of the product. Define any safeguards or 
actions that must be taken, as well as actions that must be prevented. Refer to 
any external policies or regulations that state safety issues that affect the 
product’s design or use. Define any safety certifications that must be 
satisfied.$>$

\section{Security Requirements}
$<$Specify any requirements regarding security or privacy issues surrounding use 
of the product or protection of the data used or created by the product. Define 
any user identity authentication requirements. Refer to any external policies or 
regulations containing security issues that affect the product. Define any 
security or privacy certifications that must be satisfied.$>$

\section{Software Quality Attributes}
$<$Specify any additional quality characteristics for the product that will be 
important to either the customers or the developers. Some to consider are: 
adaptability, availability, correctness, flexibility, interoperability, 
maintainability, portability, reliability, reusability, robustness, testability, 
and usability. Write these to be specific, quantitative, and verifiable when 
possible. At the least, clarify the relative preferences for various attributes, 
such as ease of use over ease of learning.$>$

\section{Business Rules}
$<$List any operating principles about the product, such as which individuals or 
roles can perform which functions under specific circumstances. These are not 
functional requirements in themselves, but they may imply certain functional 
requirements to enforce the rules.$>$


\chapter{Other Requirements}
$<$Define any other requirements not covered elsewhere in the SRS. This might 
include database requirements, internationalization requirements, legal 
requirements, reuse objectives for the project, and so on. Add any new sections 
that are pertinent to the project.$>$

\section{Appendix A: Glossary}
%see https://en.wikibooks.org/wiki/LaTeX/Glossary
$<$Define all the terms necessary to properly interpret the SRS, including 
acronyms and abbreviations. You may wish to build a separate glossary that spans 
multiple projects or the entire organization, and just include terms specific to 
a single project in each SRS.$>$

\section{Appendix B: Analysis Models}
$<$Optionally, include any pertinent analysis models, such as data flow 
diagrams, class diagrams, state-transition diagrams, or entity-relationship 
diagrams.$>$

\section{Appendix C: To Be Determined List}
$<$Collect a numbered list of the TBD (to be determined) references that remain 
in the SRS so they can be tracked to closure.$>$

\end{document}
